\documentclass[fontsize=10pt]{article}
\usepackage[utf8]{inputenc}
\usepackage[T1]{fontenc}
\usepackage{graphicx} % handles figures
\usepackage[fleqn]{mathtools}
\usepackage{amsmath}
\usepackage{hyperref}
\usepackage{amssymb}
\usepackage{xcolor}

%Insertion de tout un tas de librairie qui nous seront probablement inutiles pour la pluspart mais it's always good to have them
\title{\textbf{Maths Discrètes}\\ Solutions TP 3}
\date{}
\begin{document}
\maketitle % fais le titre écris plus haut


\section*{Exercice 1}
$s = 15$, \hspace{1cm}$d = 4$\\
$$\begin{pmatrix}
s + d -1\\
s
\end{pmatrix}
=
\begin{pmatrix}
18\\
 15
\end{pmatrix}$$
\section*{Exercice 2}
\begin{align*}
&x > 0 \rightarrow x \geq 1 \hspace{0.5cm} &;\hspace{0.5cm}&x' = x-1 \hspace{1cm}\text{ou} \hspace{0.5cm}&x = x'+1 \\
&y \geq 9 &; \hspace{0.5cm}&y' = y-9 & y = y'+9\\
&z>-2 \rightarrow z\geq -1&; \hspace{0.5cm}&z' = z+1 & z = z'-1\\
&t \geq 0 &; \hspace{0.5cm}&t' = t & t = t'\\
&u > 10 \rightarrow u \geq 11 &; \hspace{0.5cm}&u' = u-11 & u = u' +11
\end{align*}
\begin{align*}
x+y+z+t+u &= 60\\
(x'+1)+(y'+9)+(z'-1)+t'+(u'+11) &= 60\\
x'+y'+z'+t'+u' &= 40\\
\end{align*}
donc: $s = 40, d = 5$\\
$$\begin{pmatrix}
s + d -1\\
s
\end{pmatrix}
=
\begin{pmatrix}
44\\
40
\end{pmatrix}$$

\section*{Exercice 3}
\begin{enumerate}
\item $s \leq 6 $ et $s \in \mathbb{N}$ \hspace{0.5cm}, \hspace{0.5cm} d = 4
\begin{align*}
&\underset{n=0}{\overset{6}{\sum}}\begin{pmatrix}
n + 4 -1\\
4-1
\end{pmatrix} &&=
\underset{n=0}{\overset{6}{\sum}}\begin{pmatrix}
n + 3\\
3
\end{pmatrix}\\
&&&=
\underset{n=0}{\overset{6}{\sum}}\begin{pmatrix}
n + 3\\
0+3
\end{pmatrix}\\
&&&=\underset{n=0}{\overset{6}{\sum}}\begin{pmatrix}
6+1+3\\
0+1+3
\end{pmatrix}\\&&&= \begin{pmatrix} 10 \\ 4\end{pmatrix} = 210
\end{align*}
\item$ 0 < s \leq 2$ et $s \in \mathbb{Z}$, $d=4$\\
puisque $x,y,z,t \geq 1$, alors $s \leq 6 - 4(1)$.
\begin{align*}
\underset{n=0}{\overset{2}{\sum}}
\begin{pmatrix}
n+4-1\\
4-1
\end{pmatrix} &=\underset{n=0}{\overset{2}{\sum}}
\begin{pmatrix}
n+3\\
4-1
\end{pmatrix}\\
&=\underset{n=0}{\overset{2}{\sum}}
\begin{pmatrix}
n+3\\
0+3
\end{pmatrix}\\
&=
\begin{pmatrix}
6\\
4
\end{pmatrix}\\
&=15
\end{align*}

\item
\begin{alignat*}{4}
 &x \geq 3 &&; x =x' +3\\
&y \geq -1 &&; y = y' -1\\
&z \geq 1 &&; z = z' +1\\
&t \geq -2 &&; t = t'-2
\end{alignat*}
$\Rightarrow$ $(x'+3)+(y'-1)+(z'+1)+(t'-2) \leq 6$\\
$\Leftrightarrow$ $x'+y'+z'+t' \leq 5$ \\
$\Rightarrow$ $s \leq 5 \text{ et } s \in \mathbb{Z}, d=4$ \hspace{3cm} $\in \mathbb{Z}$ parce que $x,y,z,t\geq 0$\\
\begin{alignat*}{4}
\underset{n=0}{\overset{5}{\sum}}
\begin{pmatrix}
n + 4 -1\\
4-1
\end{pmatrix}
&= \underset{n=0}{\overset{5}{\sum}}
\begin{pmatrix}
n+3\\
3
\end{pmatrix}\\
&= 
\begin{pmatrix}
9\\
4
\end{pmatrix} = 126
\end{alignat*}
\end{enumerate}
\section*{Exercice 4}
$$x+y+z = 415 -t$$
$$(415-t)+u = 273$$
$$t-u = 142$$
$$\text{si on conniat $u$ on connait $t$}$$
$$\text{(et $u < t$ donc $\mathbb{Z}_+$)}$$
$$ (x'+1)+(y'+1)+(z'+1)+(u'+1)=273$$
$$ x'+y'+z'+u' = 269$$
$$\begin{pmatrix}
269+4-1\\
4-1
\end{pmatrix} = 
\begin{pmatrix}
272\\
3
\end{pmatrix}
$$
\section*{Exercice 5}
$$ (x'+1)+(y'+1)+(z'+1)+(t'+1) \leq 99$$
$$ x'+y'+z'+t' \leq 95$$
$$ \underset{n=0}{\overset{95}{\sum}}
\begin{pmatrix}
n+3\\
3
\end{pmatrix} = \begin{pmatrix}
99\\
4
\end{pmatrix}$$



\section*{Exercice 6}
Sans les U, il y a 12 lettres, avec lesquelles on peut composer $\dfrac{12!}{2!2!2!2!3!1!}$ mots (\textit{voir coefficient multinomial}).\\
Il y a 13 emplacements pour U dans un mot de 12 lettres pour qu'il n'en ait jamais 2 côte à côte: devant le mot, entre chaque lettre, à la fin du mot.\\
On a 9 U, soit $\begin{pmatrix}
13\\
9
\end{pmatrix}$ possibilités.\\
On peut donc composer $\dfrac{12!}{2!2!2!2!3!1!} \times \begin{pmatrix}
13\\
9
\end{pmatrix}$ mots.


\section*{Exercice 7}
16
\section*{Exercice 8}
[Preuve par contradiction]\\
Pour $n$ éléments, on a \\
$\rightarrow$ maximum $n$ éléments identiques\\
$\rightarrow$ maximum $n$ éléments différents\\
Pour chaque type d'élément dans $n$, on a au plus $n$ élements, soit $n^2$ objets au total.\\
Si on a $n^2+1$ objets, alors on doit avoir soit $n+1$ éléments différents soit $n+1$ objets d'un type.
\section*{Exercice 9}
\begin{enumerate}
\item $8^{16}$
\item $8^8$
\item $8^{11}$
\item $7^{14}+7^{15}+7^{16}$
\item $7^{11}+7^{10}+7^{9}$
\end{enumerate}
\section*{Exercice 10}
[Preuve par contradiction]\\
voir \textit{Ramsey number}: $R_{3\cdot3}$\\

Soit une 2-coloration (\textit{qu'on peut facilement utiliser représenter comme relations amies/ennemies}) du graphe complet à six sommets $K_6$.
Chaque sommet de ce graphe possède donc 5 arêtes. D'après le \textit{principe des tiroirs}, au moins 3 d'entre elles sont de la même couleur.\\
Soient les arêtes $\{v,r\}, \{v,s\}, \{v,t\}$ de la première couleur, pour $v,r,s,t \in K_6$.\\
Si l'une des arêtes $\{r,s\}, \{t,s\}, \{t,r\}$ est également de la première couleur, alors le triangle correspondant est entièrement de la première couleur, sinon c'est le triangle formé par ces trois dernières arêtes qui est entièrement de la deuxième couleur.

\end{document}
