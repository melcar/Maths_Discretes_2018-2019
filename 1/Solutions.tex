\documentclass[fontsize=10pt]{article}
\usepackage[utf8]{inputenc}
\usepackage[T1]{fontenc}
\usepackage{graphicx} % handles figures
\usepackage[fleqn]{mathtools}
\usepackage{amsmath}
\usepackage{hyperref}
\usepackage{amssymb}
\usepackage[makeroom]{cancel}

%Insertion de tout un tas de librairie qui nous seront probablement inutiles pour la pluspart mais it's always good to have them
\title{\textbf{TP1}\\ Solutions}
\author{Beltus Marcel}
\date{}
\begin{document}
\maketitle % fais le titre écris plus haut


\section*{Exercice 1}
\begin{enumerate}
\item $|A\times B | = |A|\cdot|B|=ab$
\item $|B^A|=b^a$
\item $|\{f:A \rightarrow B : \text{f est une projection de A dans B, injective}\}| = \frac{b!}{(b-a)!}\text{si $a \leq b$, sinon 0.}$
\item $|S(A)| = a!$
\end{enumerate}
\section*{Exercice 2}
\begin{enumerate}
\item $|F^x| = 1$, donc $|F|=1$ puisque $|F^x|=f^x$ pour $|F|=f$ et $|X|=x$ .
\item $|Y^F|=1$, donc F n'existe pas si non vide.
\end{enumerate}

\section*{Exercice 3}
\begin{enumerate}
\item \begin{align*}
\text{Supposons $f$ non-injective}&\\
&\text{alors } \exists a,b \in A \text{ tq } f(a)=f(b)\\
&\text{et donc } g\mathord\circ f (a) = g\mathord\circ f (b)\\
&\text{mais }g\mathord\circ f \text{ est injective, donc $f$ doit être inejctive}
\end{align*}
\item \begin{align*}
\text{Supposons $g$ non surjective}& \\
&\text{alors } \exists c \in C : \nexists b \in B \text{ tq } g(b) = c\\
&\text{et donc }\exists c \in C : \nexists a \in A \text{ tqt } g\mathord\circ f(a ) = c\\
&\text{mais $g\mathord\circ f$ est surjective, donc $g$ doit être surjective}
\end{align*}
\item \begin{align*}
\text{si $g\mathord\circ f$ est bijective, alors }&\text{$g\mathord\circ f$ est surjective et injective}\\
& \text{comme démontré en 1, $f$ est injective}\\
& \text{comme démontré en 2, $g$ est donc sujective}
\end{align*}
\end{enumerate}
\section*{Exercice 4}
\section*{Exercice 5}
\section*{Exercice 6}

\end{document}
